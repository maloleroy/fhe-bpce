\documentclass{article}
\usepackage[utf8]{inputenc}
\usepackage[T1]{fontenc}
\usepackage{graphicx}
\usepackage[top=2cm,bottom=2cm,left=2.5cm,right=2.5cm]{geometry}
\usepackage{amsfonts, amsmath, amssymb, ragged2e, mathrsfs, enumitem, xcolor}
\usepackage[hidelinks]{hyperref}

\title{FHE 1}
\author{Ethan Bandasack}
\date{Mars 2025}

\begin{document}

\maketitle

\setlength{\parindent}{0cm}
\setlength{\parskip}{0.2cm}

\section{Introduction}

Le système permet de traiter des entiers. Notons à ce titre $\mathcal{X} \subseteq \mathbb{N}$ l’ensemble des entrées.

Notons également $D > 0$ un paramètre du système qui servira à découper les entrées.

Une clé est un élément $S \in \mathbb{N}$ (qui n’a ici pas besoin d’être premier).

Sauf mention contraire au cours du document, $x$ et $y$ sont des éléments de $\mathcal{X}$ représentant une entrée inconnue.


\section{Chiffrement}

Commençons par décomposer $x$ en base $D$ de sorte que

\begin{equation}
\label{decomposition}
    x = \displaystyle\overline{x_{n-1}\ldots x_{2} x_{1} x_{0}}^{(D)} = \sum_{i=0}^{n-1}x_{i} D^i
\end{equation}

Le chiffrement se compose de la façon suivante :
\begin{itemize}
    \item On génère un bruit aléatoire $e \in (\mathbb{N}^*)^n$
    \item On chiffre $x$ en $\mathcal{C}(x, S) = \Big( S \times e_i + x_i \Big)_{0\le i < n}$
\end{itemize}

\subsection{Hypothèse 1}
\label{hypothese1}
On considère que $n$ est assez grand pour que $x$, et tous les autres nombres considérés, soient représentables dans cette base, c’est-à-dire que $x_i \in \{0, \ldots, D^n-1\}$.

\section{Déchiffrement}

Si $S\ge D$, alors $\mathcal{S}(x, S)$ peut être déchiffré en prenant la division euclidienne de chaque composante par $S$. On obtient alors $(x_i)_{(0\le i<n)}$ (en effet, tous les $x_i$ sont compris entre 0 et $D-1$), ce qui donne immédiatement $x$ si on connaît la base $D$ (voir \ref{decomposition}).

On note d'ailleurs qu'en notant $P_x=\displaystyle\sum_{i=0}^{n-1}x_i X^i$, on a $x=P_x(D)$. On gardera cette notation pour la suite.

On va d'ailleurs considérer que $\mathcal{C}(x, S)$ est un polynôme de $\mathbb{Z}_D[X]$ à coefficients entiers positifs entre 0 et $D-1$.

\section{Propriétés homomorphes}

\subsection{Addition}
\label{addition}
La propriété d'homomorphie pour l'addition entre deux éléments de $\mathcal{X}$ est assurée si $S\ge 2D-1=(D-1)+(D-1)+1$.

\subsubsection{Démonstration}
Si $S\ge 2D-1$, alors pour $x, y \in \mathcal{X}$, on a :

\begin{equation}
\mathcal{C}(x, S) + \mathcal{C}(y, S) = \displaystyle \sum_{i=0}^{n-1} \Big( S(e_i^{(x)} + e_i^{(y)}) + x_i + y_i \Big) X^i= \mathcal{C}(x+y, S)
\end{equation}

car pour tout $i$, $0\le x_i + y_i \le 2D-2<S$.


\subsection{Multiplication}
\label{multiplication}
La propriété d'homomorphie pour la multiplication entre deux éléments de $\mathcal{X}$ est assurée si $S\ge D^2-2D+2=(D-1)(D-1)+1$. 

\subsubsection{Démonstration}
Si $S\ge D^2-2D$, alors pour $x, y \in \mathcal{X}$, on a :

\begin{align}
\mathcal{C}(x, S)\mathcal{C}(y, S) &= \displaystyle \sum_{i=0}^{n-1} \sum_{j=0}^{n-1} \Big( Se_i + x_i \Big) \Big( Se_j + y_j \Big) X^{i+j}\\
&=\displaystyle \sum_{i=0}^{2n-2} \sum_{\substack{0\le j,k < n\\ j+k=i}}\Big( S^2 e_j e_k + S(e_j y_k + x_j e_k) + x_j y_k \Big) X^{i}\\
&= \mathcal{C}(\displaystyle \sum_{i=0}^{2n-2} \sum_{\substack{0\le j,k < n\\ j+k=i}} x_jy_k D^i, S)
\end{align}

Le passage à la dernière ligne se justifie car, pour tout $i$, $0\le x_i y_i \le (D-1)^2 < D^2-2D+2 \le S$.

On remarque de plus que $\displaystyle \sum_{i=0}^{2n-2} \sum_{\substack{0\le j,k < n\\ j+k=i}} x_jy_k D^i = \sum_{i=0}^{n-1}\sum_{j=0}^{n-1} x_jy_k X^{j+k} = P_x(D)P_y(D)=xy$.

Enfin, d'après l'hypothèse \ref{hypothese1}, on a peut arrêter la somme à $n-1$, ce qui montre d'ailleurs que $P_{xy}(D)=P_x(D)P_y(D)$.

On a bien $\mathcal{C}(x, S)\mathcal{C}(y, S) = \mathcal{C}(xy, S)$.


\subsection{Fonction polynomiale}

On déduit de la partie \ref{addition} que, par récurrence, $\mathcal{C}(kx, S)= k\mathcal{C}(x, S)$ pour tout $k\in\mathbb{N}$ (à condition que $S\ge k(D-1)+1$).

On en déduit de la même façon que la propriété d'homomorphie est assurée pour une somme de $k\in\mathbb{N}^*$ éléments de $\mathcal{X}$ si $S\ge k(D-1)+1$.

Idem pour la multiplication de $k\in\mathbb{N}^*$ éléments de $\mathcal{X}$ si $S\ge (D-1)^k+1$.

En ajoutant la partie \ref{multiplication}, on peut généraliser cette propriété à toute fonction polynomiale à plusieurs variables à coefficients entiers positifs $P:\mathbb{N}^k\longrightarrow\mathbb{N}$ du moment que $S>P(D-1,\dots, D-1)$.

\subsubsection{Démonstration}

Soit $P:\mathbb{N}^k\longrightarrow\mathbb{N}$ une fonction polynomiale à plusieurs variables à coefficients entiers positifs. On note donc $P(x_1, \dots, x_k) = \displaystyle\sum_{\substack{i_1, \dots, i_k \ge 0\\ \prod_j i_j \le d}} a_{i_1, \dots, i_k} \prod_{j=1}^kx_j^{i_j}$, avec $a_{i_1, \dots, i_k} \in \mathbb{N}$ et $d$ le degré de $P$.

Supposons que $S>P(D-1,\dots, D-1)$. Alors pour $(x_1, \dots, x_k)\in\mathcal{X}^k$, on applique d'abord la propriété d'homomorphie pour la multiplication de $\displaystyle\sum_{j=1}^k i_j$ éléments de $\mathcal{X}$ :
\begin{equation}
    P(\mathcal{C}(x_1, S), \dots, \mathcal{C}(x_k, S)) = \displaystyle\sum_{\substack{i_1, \dots, i_k \ge 0\\ \prod_j i_j \le d}} a_{i_1, \dots, i_k} \prod_{j=1}^k\mathcal{C}(x_j, S)^{i_j}=\displaystyle\sum_{\substack{i_1, \dots, i_k \ge 0\\ \prod_j i_j \le d}} a_{i_1, \dots, i_k} \mathcal{C}\Big(\prod_{j=1}^kx_j^{i_j}, S\Big)
\end{equation}

car pour tous $i_1, \dots, i_k$, $0\le\displaystyle\prod_{j=1}^kx_j^{i_j} \le (D-1)^{\sum_{i=1}^k i_j} \le (D-1)^d \le P(D-1,\dots, D-1) < S$.

\textit{Ce n'est vrai que si le coefficient correspondant $a_{i_1, \dots, i_k}$ est non nul, dans le cas contraire cela n'a aucune incidence sur $P(\mathcal{C}(x_1, S), \dots, \mathcal{C}(x_k, S))$.}

On applique ensuite la propriété d'homomorphie pour l'addition de $\displaystyle\sum_{\substack{i_1, \dots, i_k \ge 0\\ \prod_j i_j \le d}} a_{i_1, \dots, i_k}$ éléments de $\mathcal{X}$ :
\begin{equation}
    P(\mathcal{C}(x_1, S), \dots, \mathcal{C}(x_k, S)) = \displaystyle\sum_{\substack{i_1, \dots, i_k \ge 0\\ \prod_j i_j \le d}} a_{i_1, \dots, i_k} \mathcal{C}\Big(\prod_{j=1}^kx_j^{i_j}, S\Big)=\mathcal{C}\Big(P(x_1, \dots, x_k), S\Big)
\end{equation}

car pour tous $i_1, \dots, i_k$, $0\le \displaystyle \sum_{\substack{i_1, \dots, i_k \ge 0\\ \prod_j i_j \le d}} a_{i_1, \dots, i_k} \prod_{j=1}^kx_j^{i_j} \le \displaystyle \sum_{\substack{i_1, \dots, i_k \ge 0\\ \prod_j i_j \le d}} a_{i_1, \dots, i_k} (D-1)^d \le P(D-1,\dots, D-1) < S$.

\end{document}